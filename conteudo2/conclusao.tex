\chapter{Conclusão}
Ao longo desse trabalho, a importância do desenvolvimento de soluções baratas e eficientes para a detecção e diagnóstico de doenças na produção agrícola ficou evidente. A análise da situação atual dos conjuntos de dados abertos e das aplicações baseadas neles revelou limitações significativas, como problemas de qualidade e uma relativa escassez de dados, que comprometem a eficácia dessas iniciativas.

Então, fica claro que o desenvolvimento dessas soluções exige a criação de conjuntos de dados abertos, livres e robustos capazes de sustentar o desenvolvimento de aplicações voltadas ao diagnóstico e à detecção de doenças em plantas por meio de imagens. Essa abordagem beneficia os agricultores e a comunidade científica em geral.

O objetivo deste trabalho de formatura foi justamente implementar uma solução promissora para essa demanda: o sistema \emph{TomatoHealth}. Um sistema que oferece uma interface intuitiva para que agricultores diagnostiquem doenças em suas plantas --- inicialmente voltada para tomateiros, mas com potencial de expansão para outras culturas. Para efetuar essa tarefa, o \emph{TomatoHealth} conta com um modelo de detecção de objetos (\emph{YOLOv8}) que inicialmente foi treinado com um subconjunto do \emph{dataset PlantDoc}.

O diferencial do \emph{TomatoHealth} está na possibilidade de melhoria contínua do conjunto de dados. As imagens enviadas pelos usuários passam pela revisão de especialistas e, então, são incorporadas ao conjunto de dados original. Ainda, podemos utilizar o novo conjunto de dados para retreinar o modelo de detecção de objetos e melhorar o desempenho da ferramenta de diagnóstico do \emph{TomatoHealth}. Esse processo melhora o desempenho do diagnóstico e contribui para a construção de um \emph{dataset} mais representativo das condições reais, já que utiliza fotos capturadas por dispositivos móveis diretamente em campo.

Por fim, este trabalho busca transformar a forma como os conjuntos de dados são produzidos, promovendo o desenvolvimento de soluções mais acessíveis e abertas. Essa abordagem, além de democratizar o acesso à tecnologia, visa expandir o alcance do conhecimento e impulsionar o progresso científico de maneira inclusiva.
